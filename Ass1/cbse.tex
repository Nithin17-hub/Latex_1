\documentclass{article}
\usepackage{amsmath, amssymb, tabularx}
\usepackage{enumitem}
\usepackage{tfrupee}
\usepackage{mathabx}
\begin{document}
\section*{\centering MATHEMATICS}
\section*{\centering SECTION-A}
\subsection*{\centering Question numbers 1 to 6 carry 1 mark each.}
\begin{enumerate}
	\item Write the value of $\left|
		\begin{matrix}
		a-b&b-c&c-a\\
		b-c&c-a&a-b\\
		c-a&a-b&b-c\\
		\end{matrix}
			\right|$.
	\item If $A$=$\left[
			\begin{matrix}
				1&-2&3\\
				-4&2&5\\
			\end{matrix}
		\right]$ and $B$=$\left[
		\begin{matrix}
			2&3\\
			4&5\\
			2&1\\
		\end{matrix} \right]$ and $BA$=(bij),find $b_{21}$+$b_{32}i$.
\item Write the number of all the possible matrices of order $2\times3$ with each entry $1$ or $2$.
\item Write the coordinates of the point which is the reflection of the point $(\alpha,\beta,\gamma)$ in the XZ-plane.
\item Find the position vector of the point which divides the join of points with position vectors $\overset{\rightarrow}{a}+3\overset{\rightarrow}{b} and \overset{\rightarrow}{a}-\overset{\rightarrow}{b}$ in the ratio 1:3.
\item If $|\vec{a}|$=$4$,$|\vec{b}|$=$3$ and $\vec{a}.\vec{b}$=$6\sqrt{3}$,then find the value of $|\vec{a}\times\vec{b}|$. 
\section*{\centering SECTION-B}
\subsection*{\centering Question numbers 7 to 19 carry 4 marks each.}
\item Solve for $x:tan^{-1}\left(\frac{2-x}{2+x}\right)$=$\left(\frac{1}{2}\right)tan^{-1}\left(\frac{x}{2}\right)$,$x \textgreater 0$.
	\subsection*{\centering OR}
		Prove that $2sin^{-1}\left(\frac{3}{5}\right)-tan^{-1}\left(\frac{17}{31}\right)=\left(\frac{\pi}{4}\right)$.
\item On her birthday Seema decided to donate some money to chldren of an rophanage home.If there was 8 children less,everyone would have go \rupee $10$ more.However,if there were $16$ children more,every one would have got\rupee $10$ less.Using matrix method,find the numbers of children and the amount distributed by Seema.What values are refleced by Seema's decision?
\item If $x$=$e^{cos2t}$ and $y$=$e^{sin2t}$,prove that $\frac{dy}{dx}$=$\frac{-y}{x}$ $\frac{logx}{logy}$.\\ Verify mean value theorem for the function $f(x)=2sinx+sin2x$ on $[o,x]$.
\item Find the eqation of the tangent line to the curve $y$=${\sqrt5x-3}{-5}$,when parallel to the line $4x-2y=5$=$0$.
\item Show that the function $f$ given by fx=$\begin{cases}\frac{e^{\frac{1}{x}}-1}{e^{\frac{1}{x}}+1},& if x\neq0\\
$-1$,& if x=0 \end{cases}$ \\is discontinuous at $x$=$0$. 
\item Evaluate: $\int_{1}^{5}$ $\{|x-1| + |x-2| + |x-3|\}$dx.
		\subsection*{\centering OR}
		Evaluate: $\int_{0}^{\pi} \frac{xsinx}{1+3cos^2x}dx$.
	\item Find: $\int\frac{2x+1}{(x^2)(x^2)}$dx.
	\item Find $\int (3x+5)$ $\sqrt{5x+4x-2x^2}$dx.
	\item x$\frac{dy}{dx}+y-x+xycotx=0$;$x\neq0$.
	\item Solve the different equation:$(x^2+3xy+y^2)dx$ $-x^2dy=0$ given that $y=0$,when $x=1$.
	\item Find the angle between the vectors $\vec{a}+\vec{b}$ and $\vec{a}-\vec{b}$ if $\vec{a}=2\hat{i}-\hat{j}+3\hat{k}$ and $\vec{b}=3\hat{i}+\hat{j}-2\hat{k}$,and hence find a vector perpendicular to both $\vec{a}+\vec{b}$ and $\vec{a}-\vec{b}$.
	\item Show that the line $\frac{x-1}{3}=\frac{y-1}{-1}=\frac{z+1}{0}$ and $\frac{x-4}{2}=\frac{y}{0}=\frac{z+1}{3}$ intersect.Find their point of intersection.
	\item A committee of $4$ students is selected at random from a group consisting of $7$ boys and $4$ girls. Find the probability that there are exactly $2$ boys in the committee, given that at least one girl must be there in the committee.
		\subsection*{\centering OR}
		A random variable $X$ has the following probability distribution:\\
\begin{table}[htb]
\centering
\begin{tabularx}{\textwidth}{|c|X|X|X|X|X|X|X|}
\hline
X & 0 & 1 & 2 & 3 & 4 & 5 & 6 \\
\hline
P(X) & c & 2c & 2c & 3c & $c^2$ & $2c^2$ & $7c^2$+c \\
\hline
\end{tabularx}
\end{table}\\		
			Find the value of $C$ and also calculate mean of the distribution.
		\section*{\centering SECTION C}
		\subsection*{\centering Question numbers 20 to 26 carry 6 marks each.}
	\item Show that the relation $R$ defined by (a,b)R(c,d)$\Rightarrow a+d=b+c$ on the $A\times A$,where $A=\{1,2,3,...10\}$ is an equivalence class $[(3,4)];a,b,c,d \epsilon A$.
		\item Solve for x: $\left|
		\begin{matrix}
                 a+x&a-x&a-x\\
	         a-x&a+x&a-x\\
		 a-x&a-x&a+x\\
		 \end{matrix} \right|$=$0$,\\
		 using properties of determinants.
		 \subsection*{\centering OR}
		 Using elementary row operation find the inverse of matrix $X$ $A$=$\left[ 
		 \begin{matrix}
			 3&-3&4\\
			 2&-3&4\\
			 0&-1&1\\
		 \end{matrix}
		 \right]$ and hence solve the following system of equations $3x-3y+4z=21,2x-3y+4z=20,-y+z=5$.
	 \item Show that height of the cylinder of greatest volume which can be in scribed in aright circular cone of height h and semi-vertical angle $\alpha$ is one third that of and greatest volume of cylinder is $\frac{4}{27}\pi \tan^2 \alpha$.
		 \subsection*{\centering OR}
		 Find the intervals in which the function $f(x)=\frac{4sinx}{2+cosx}-x;0\leq x \leq 2\pi$ is strictly increasing or strictly decreasing.
	 \item Using integration,find the area of the triangle formed by inegative x-axis and tangent and normal to the circle $x^2+y^2=9$ at $(-1,2\sqrt{2})$.
	 \item Find the coordinates of the foot of perpendicular distance frm the point $P(4,3,2)$ to the plane $x+2y+3z=2$.Also find the image of $P$ in the plane.
	 \item $A$, $B$ and $C$ throw a pair of dice in that order alternately till one of them gets a total of $9$ and wins the game. Find their respective probabilities of winning, if $A$ starts first.
	 \item A company manufactures two types of cardigans type $A$ and type $B$. It costs \rupee $360$ to make a type A cardigan and \rupee $120$ to make a type $B$ cardigan. The company can make at most $300$ cardigans and spend at most \rupee $72,000$ a day. The number of cardigans of type $B$ cannot exceed the number of cardigans of type $A$ by more than \rupee $200$. The company makes a profit of \rupee $100$ for each cardigan of type $A$ and \rupee $50$ for every cardigan of type $B$.

Formulate this problem as a linear programming problem to maximise the profit to the company. Solve it graphically and find maximum profit.
\end{enumerate}
\end{document}

